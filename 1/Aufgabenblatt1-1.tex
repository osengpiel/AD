\documentclass{article}

\usepackage[utf8]{inputenc} \usepackage[ngerman]{babel}

\usepackage{amssymb} \usepackage{amsmath}

\usepackage{latexsym}

\title{Aufgabenblatt 1 - Aufgabe 1}

\author{Oliver Sengpiel \\
        Laura Schüttpelz \\
        Merlinde Claudia Tews}

\begin{document}

\maketitle

% blablabla
\begin{enumerate}
\item[(a)]
    \begin{align*}
        &\frac{1}{n} \prec 1 \prec \log \log n \prec \log n \asymp \log n^3 \prec
        \log n^{\log n} \prec n \log n \\
        &\text{und} \quad n \log n \prec n^{0.01} \prec \sqrt{n} \prec n^8
        \prec 2^n \prec 8^n \prec n^n \prec n!
    \end{align*}


\underline{Beweise:}\\
Zunächst gilt es zu beweisen, dass
\[
    \frac{1}{n} \in o(1) \Leftrightarrow \lim_{n \to \infty}
    \frac{\left( \frac{1}{n} \right)}{1} = 0
\]
gilt. Der Beweis ist trivial, denn es ist
\begin{align*}
    &\lim_{n \to \infty} \frac{\left( \frac{1}{n} \right)}{1} \\
    \Leftrightarrow &\lim_{n \to \infty} \frac{1}{n} = 0 \quad \quad \square
\end{align*}
und daraus folgt, dass $\frac{1}{n} \prec 1$ gilt.\\
Analog dazu sind die folgenden Beweise.

\underline{$1 \prec \log \log n$:}
\[
    \frac{1}{\log \log n} \underset{n \to \infty}{\longrightarrow} 0 \Rightarrow
    1 \prec \log \log n \qquad \square
\]

\underline{$\log \log n \prec \log n$:}
Hier muss noch ein Beweis rein.

\underline{$\log n \asymp \log n^3$:}\\
$\log n \in \mathcal{O}$ gilt genau dann, wenn $0 < \lim_{n \to \infty} \frac{\log
n}{\log n^3} < \infty$ gilt. Für den Grenzwert von $\frac{\log n}{\log n^3}$ für
$n \to \infty$ gilt:
\[
    \lim_{n \to \infty} \frac{\log n}{\log n^3} = \frac{\log n}{3 \cdot \log n}
    = \frac{1}{3}
\]
woraus folgt, dass die Behauptung wahr ist.

\item[(b)]

\begin{enumerate}
\item[(i)]
blub
\item[(ii)]
blubblubb
\item[(iii)]
blubblubb
\end{enumerate}

\end{enumerate}
\end{document}
