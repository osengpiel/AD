

\documentclass{article}

\usepackage[utf8]{inputenc} \usepackage[ngerman]{babel}

\usepackage{amssymb} \usepackage{amsmath}

\usepackage{latexsym}

\title{Aufgabenblatt 1 - Aufgabe 1}
\author{}

\begin{document}

\maketitle

% blablabla
\begin{enumerate}
\item[(a)]
    \begin{align*}
        &\frac{1}{n} \prec 1 \prec \log \log n \prec \log n \asymp \log n^3 \prec
        \log n^{\log n} \prec n \log n \\
        &\text{und} \quad n \log n \prec n^{0.01} \prec \sqrt{n} \prec n^8
        \prec 2^n \prec 8^n \prec n^n \prec n!
    \end{align*}


\underline{Beweise:}\\
Zunächst gilt es zu beweisen, dass
\[
    \frac{1}{n} \in o(1) \Leftrightarrow \lim_{n \to \infty}
    \frac{\left( \frac{1}{n} \right)}{1} = 0
\]
gilt. Der Beweis ist trivial, denn es ist
\begin{align*}
    &\lim_{n \to \infty} \frac{\left( \frac{1}{n} \right)}{1} \\
    \Leftrightarrow &\lim_{n \to \infty} \frac{1}{n} = 0 \quad \quad \square
\end{align*}
und daraus folgt, dass $\frac{1}{n} \prec 1$ gilt.\\
Analog dazu sind die folgenden Beweise.

\underline{$1 \prec \log \log n$:}
\[
    \frac{1}{\log \log n} \underset{n \to \infty}{\longrightarrow} 0 \Rightarrow
    1 \prec \log \log n \qquad \square
\]

\underline{$\log \log n \prec \log n$:} \\
Hier muss noch ein Beweis rein.

\underline{$\log n \asymp \log n^3$:}\\
$\log n \in \mathcal{O}(\log n^3)$ gilt genau dann, wenn $0 < \lim_{n \to \infty} \frac{\log
n}{\log n^3} < \infty$ gilt. Für den Grenzwert von $\frac{\log n}{\log n^3}$ für
$n \to \infty$ gilt:
\[
    \lim_{n \to \infty} \frac{\log n}{\log n^3} = \frac{\log n}{3 \cdot \log n}
    = \frac{1}{3}
\]
woraus folgt, dass die Behauptung wahr ist.

\underline{$\log n^3 \prec \log n^{\log n}$:} \\
\begin{align*}
    \lim_{n \to \infty} \frac{\log n^3}{\log n^{\log n}} &= \lim_{n \to \infty}
        \frac{3 \cdot \log n}{\log n \cdot \log n} \\
        &= \lim_{n \to \infty} \frac{3}{\log n} \\
        &= 0 \qquad \square
\end{align*}

\underline{$\log n^{\log n} \prec n \log n$:} \\
\begin{align*}
    \lim_{n \to \infty} \frac{\log n^{\log n}}{n \log n} &= \lim_{n \to \infty}
        \frac{\log n \cdot \log n}{n \log n} \\
        &= \lim_{n \to \infty} \frac{\log n}{n} \qquad mit ~ l'Hospital \\
        &= 0 \qquad \square
\end{align*}

\underline{$n \log n \prec n^{0.01}$:} \\
\begin{align*}
    \lim_{n \to \infty} \frac{n \log n}{n^{0.01}} &= \lim_{n \to \infty}
        \frac{n \log n} \qquad l'Hospital \\
        &= \lim_{n \to \infty} \frac{\frac{1}{n}}{\frac{1}{n^{0.99}}} \\
        &= \lim_{n \to \infty} \frac{n^{0.99}}{n} \\
        &= 0 \qquad \square
\end{align*}

\underline{$n^{0.01} \prec n^{\frac{1}{2}}$:} \\
Da Polynome mit einem höheren Grad immer schneller Wachsen, als
Polynome mit einem niedrigeren, folgt die Behauptung. $\quad \square$

\underline{$n^{\frac{1}{2}} \prec n^8$:} \\
Siehe vorheriger Beweis. $\quad \square$

\underline{$n^8 \prec 2^n$:} \\
Hier kommt noch ein Beweis hin.

\underline{$2^n \prec 8^n$:} \\
Hier kommt noch ein Beweis hin.

\underline{$8^n \prec n!:$} \\
Für $n \to n+1$ wird $n^8$ mit 8 multipliziert, $n!$ jedoch mit $n+1$. Deshalb
steigt $n!$ auf Dauer stärker als $n^8$. (Beweis kann durch Induktion gemacht
werden).

\underline{$n! \prec n^n$:} \\
Alle einzelnen Faktoren, außer dem $n$-ten sind bei $n!$ kleiner als bei $n^n$,
weshalb letztere Funktion schneller steigt.

\item[(b)]

\begin{enumerate}
\item[(i)]
	\begin{align*}
	\text{Behauptung: } &\text{Für beliebige } b > 1 \text{ gilt: } log_b(n) \in \Theta(log_2n) \\
	&\Leftrightarrow 0 < lim_{n \rightarrow \infty}~ inf \frac{log_b(n)}{log_2(n)} \le 
		lim_{n \rightarrow \infty}~ sup \frac{log_b(n)}{log_2(n)} < \infty \\
	\text{Für } &b=2 \text{ ist } 0 <  lim_{n \rightarrow \infty}~ inf \frac{log_b(n)}{log_2(n)} 
		= lim_{n \rightarrow \infty}~ sup \frac{log_b(n)}{log_2(n)} = 1 < \infty \\
	\text{Für } &b>2 \text{ ist } log_b(n) = \frac{log_2(n)}{log_2(b)} \\
	&\Leftrightarrow 0 < lim_{n \rightarrow \infty}~ inf \frac{log_2(n)}{log_2(n)*log_2(b)} 
		\le lim_{n \rightarrow \infty}~ sup \frac{1}{log_2(b)} \\
	\text{Aber: } &lim_{n \rightarrow \infty}~ inf \frac{1}{log_2(b)} = 0  \\
	\text{Also: } &\text{Für beliebige } b > 2 \text{ gilt: } log_b(n) \notin \Theta(log_2n) \\   
	\end{align*}

\item[(ii)]
	\begin{align*}
		\text{Behauptung: } f \in O(g) &\Rightarrow g \in \omega(f) \\
		&\Leftrightarrow lim_{n \rightarrow \infty} sup \frac{f(n)}{g(n)} 
			< \infty \Rightarrow lim_{n \rightarrow \infty} inf 
			\frac{f(n)}{g(n)} \Leftrightarrow g \in o(f) \\
		&\Leftrightarrow lim_{n \rightarrow \infty} sup \frac{f(n)}{g(n)} 
			< \infty \Rightarrow lim_{n \rightarrow \infty} sup
                        \frac{f(n)}{g(n)} = 0 \\
		&\Leftrightarrow lim_{n \rightarrow \infty} sup \frac{f(n)}{g(n)} = 0 
			< \infty 
	\end{align*}

\item[(iii)]
	\begin{align*}
	\text{Behauptung: } &f_c(n) := \sum_{i=0}^n c^i: f_c(n) \in \Theta(n) \Leftrightarrow c=1 \\
	&0 < lim_{n \rightarrow \infty} inf \frac{f_c(n)}{n} \le lim_{n \rightarrow \infty} sup \frac{f_c(n)}{n} < \infty \\
	&0 < lim_{n \rightarrow \infty} inf \frac{\sum_{i=0}^n c^i}{n} \le lim_{n \rightarrow \infty} sup \frac{\sum_{i=0}^i c^i}{n} < \infty \\
	&\text{Für } c > 1 \text{geht } \frac{f_c(n)}{n} \text{gegen } \infty \text{, für } c > 0 \text{geht }  \frac{f_c(n)}{n} \text{gegen 0.} \\
	&\text{Also muss } c = 1 \text{ genau dann, wenn } f_c(n) \in \Theta(n).
	\end{align*}
\end{enumerate}


\end{enumerate}
\end{document}
