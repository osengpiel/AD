\documentclass{article}

\usepackage[utf8]{inputenc} \usepackage[ngerman]{babel}

\usepackage{amssymb} \usepackage{amsmath}

\usepackage{latexsym}

\title{Aufgabenblatt 1 - Aufgabe 1}

\author{Oliver Sengpiel \\
        Laura Schüttpelz \\
        Merlinde Claudia Tews}

\begin{document}

\maketitle


% blablabla
\begin{enumerate}
\item[(a)]

\begin{align*}
\text{Behauptung: }&B = F_n \ge 2^{0.5n} \\
\text{Induktionsanfang: } &n= 6. F_n = 5 \ge 8 = 2^3 = 2^{0.5n} \\
\text{Induktionsannahme: } &~Gelte~ B~ f"ur~ n. \\
\text{Induktionsschritt: } &F_{n+1} \ge 2^{0.5*(n+1)} \\
&F_n + F_{n-1} \ge 2^{0.5*n} * 2^{0.5} \\
&F_{n-1} \ge 2^{0.5} \text{, da } F_n \text{ laut Induktionsannahme } \ge 2^{0.5*n}
\end{align*}
	

\item[(b)]
\begin{align*}
\text{Behauptung: } &B = F_n \le 2^{c \cdot n} \\
\text{Induktionsanfang: } & n = 0. F_n = 0 \le 2^{c \cdot 0} = 1 \\
\text{Induktionsannahme: } &\text{Gelte B für n. } \\
\text{Induktionsschritt: Zu zeigen: } F_{n+1} &\le 2^{c \cdot (n+1)} \\
\text{Wenn } F_n + F_{n-1} \le 2^{c \cdot n} + 2^{c \cdot (n-1)} &\le 2^{c \cdot (n+1)}, \\
\text{dann gilt auch } F_{n+1} = F_n + F_{n-1} &\le 2^{c \cdot (n+1)} \\
\text{Zu zeigen: }  2^{c \cdot n} + 2^{c \cdot (n-1)} &\le 2^{c \cdot (n+1)} \\
2^{c^n} + 2^{c \cdot n - c \cdot 1} &\le 2^{c^n} \cdot 2^c \\
2^{c^n} + 2^{c \cdot n} \cdot 2^{c^{-1}} &\le 2^{c^n} \cdot 2^c \\
1 + \frac{1}{2^c} &\le 2^c \\
\Rightarrow c &\ge \frac{log(1+\sqrt(5))}{log(2)} -1 \\
\Rightarrow c &\ge \approx 0.7 \\
\text{Also gilt } &F_n \le 2^{c \cdot n} \text{ für } c \ge 0.7.
\end{align*}
\end{enumerate}


\end{document}
