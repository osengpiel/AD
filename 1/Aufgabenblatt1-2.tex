\documentclass{article}

\usepackage[utf8]{inputenc} \usepackage[ngerman]{babel}

\usepackage{amssymb} \usepackage{amsmath}

\usepackage{latexsym}

\title{Aufgabenblatt 1 - Aufgabe 1}

\author{Oliver Sengpiel \\
        Laura Schüttpelz \\
        Merlinde Claudia Tews}

\begin{document}

\maketitle

\begin{enumerate}
\item[(a)]


\documentclass{article}

\usepackage[utf8]{inputenc} \usepackage[ngerman]{babel}

\usepackage{amssymb} \usepackage{amsmath}

\usepackage{latexsym}

\title{Aufgabenblatt 1 - Aufgabe 1}

\begin{document}

\maketitle

% blablabla
\begin{enumerate}
\item[(a)]

\begin{align*}
\text{Behauptung: }&B = F_n \ge 2^{0.5n} \\
\text{Induktionsanfang: } &n= 6. F_n = 5 \ge 8 = 2^3 = 2^{0.5n} \\
\text{Induktionsannahme: } &~Gelte~ B~ f"ur~ n. \\
\text{Induktionsschritt: } &F_{n+1} \ge 2^{0.5*(n+1)} \\
&F_n + F_{n-1} \ge 2^{0.5*n} * 2^{0.5} \\
&F_{n-1} \ge 2^{0.5} \text{, da } F_n \text{ laut Induktionsannahme } \ge 2^{0.5*n}
\end{align*}
	

\item[(b)]


\end{enumerate}


\end{document}
